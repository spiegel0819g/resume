% !TEX program = xelatex

\documentclass{resume}
%\usepackage{zh_CN-Adobefonts_external} % Simplified Chinese Support using external fonts (./fonts/zh_CN-Adobe/)
%\usepackage{zh_CN-Adobefonts_internal} % Simplified Chinese Support using system fonts

\begin{document}
\pagenumbering{gobble} % suppress displaying page number

\name{Bo Syun, Ke}
 \basicInfo{
  \email{spiegel0819g@gmail.com} \textperiodcentered\ 
  \phone{(+886) 905-620-301}}

\section{\faUsers\ Work experience}
\datedsubsection{\textbf{ASUSTeK Computer Inc.}, Taipei, R.O.C}{2017 -- Present}
\textit{Software Engineer} in AMACS (AI Machine learning And Cloud Software center)
\datedsubsection{\textbf{ASUSTeK Computer Inc.}, Taipei, R.O.C}{2016 -- 2017}
\textit{Android Software Developer} in AMAX (ASUS Mobile Application eXperience senter)

\section{\faProductHunt\ Project}
\datedsubsection{\textbf{Computer Vision projects}}{2017 -- Present}
\role{\underline{AI Cam}} {}
\begin{description}
  \item Brief introduction:\\
Collaborate with AI Cam team. We help developing object detection feature.
\item Achievement:
\begin{itemize}
  \item it take about 60~200ms to recognize object with qualcomm snapdragon 625 and 
\end{itemize}
\item Task description:
\begin{itemize}
  \item Analyzed different models performance on datasets
  \item Developed demo app with Tensorflow on android platform
\end{itemize}
\end{description}
\role{\underline{Home Surveillance}} {}
\begin{description}
\item Brief introduction:\\
Import Face detection in door security system of company. Employee can enter office and clock in/out directly without using identification card.\\
\item Achievement:
\begin{itemize}
  \item Enhance 15\% accuracy by adding padding and not changing the ratio of input picture on 6000 people data
\end{itemize}
\item Task description:
\begin{itemize}
  \item Collected data and data cleansing
  \item Integrated the current door security system and Face detection server
  \item Optimized model with Graph transform tool, SNPE sdk
  \item Implemented KNN algorithm to classify face embeddings
  \item Analyzed Movidius NCS's performance for Machine learning
  \item Implemented different framework on FaceNet
\end{itemize}
\end{description}
\role{\underline{Detect Wallpaper}} {}
\begin{description}
\item Brief introduction:\\
In order to solve that font of apps can not be seen clearly on ZenUI desktop wallpaper. It is an app that can adjust font color automatically according to colors of ZenUI desktop wallpaper. Detect Wallpaper can choose better font color for 93\% wallpapers.\\
\item Achievement:
\begin{itemize}
  \item The accuracy of using ML to predict font is more 15\% than using analysis of lightness distribution 
\end{itemize}
\item Task description:
\begin{itemize}
  \item Studied and designed model with Tensorflow, Keras
  \item Developed app with Tensorflow on android platform 
\end{itemize}
\end{description}

\datedsubsection{\textbf{Android projects}}{2016 -- 2017}
\role{ZenUI Launcher} {https://goo.gl/vBQkv1}
\begin{description}
\item Brief introduction:\\
ZenUI is a front-end touch interface developed by ASUS with partners, featuring a full touch user interface. It is used by Asus for Android phones and tablet computers.\\
\item Task description:
\begin{itemize}
  \item Maintained AppLock
  \item Implemented Fingerprint Unlock and Pattern Unlock feature
  \item Developed different size of layouts for Android tablet or phone
\end{itemize}
\end{description}

\section{\faCogs\ Skills}
\begin{itemize}[parsep=0.5ex]
  \item Programming Languages: C, Java, Android, Python
  \item Platform: Linux
  \item Machine Learning: Tensorflow, SNPE, Movidius, Keras, OpenCV
  \item Tool: Git, Docker
\end{itemize}

\section{\faGraduationCap\ Education}
\datedsubsection{\textbf{National Chiao Tung University (NCTU)}, HsinChu City, R.O.C}{2013 -- 2015}
\textit{Master student} in Network Engineering
\datedsubsection{\textbf{National Cheng Kung University (NCKU)}, Tainan City, R.O.C}{2009 -- 2013}
\textit{B.S.} in Mathematics

\section {\faTasks\ Thesis}
\role {Bo-Syun Ke, "A QoS-aware Routing Algorithm for SDN-based Data Center Networks", Master Thesis, Department of Computer Science, National Chiao Tung University} {}

\end{document}
